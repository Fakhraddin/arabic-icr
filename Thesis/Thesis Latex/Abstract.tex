%%% LyX 2.0.5.1 created this file.  For more info, see http://www.lyx.org/.
%%% Do not edit unless you really know what you are doing.
%\documentclass[english]{report}
%\usepackage[T1]{fontenc}
%\usepackage[latin9]{inputenc}
%\setcounter{secnumdepth}{3}
%\setcounter{tocdepth}{3}
%\usepackage{setspace}
%\onehalfspacing
%\usepackage{babel}
%\begin{document}

\chapter*{Abstract}

Automated recognition of text has been an actively researched since the early days of computers. A 1972 survey cites nearly 130 works on the subject. Handwriting recognition is a task of transforming a language represented in its spatial form of graphical marks into its symbolic representation. On-line Hand Writing Recognition refers to the situation where the recognition is performed concurrently to the writing process. After a long period of focus on western and East Asian scripts there is now a general trend in the on-line handwriting recognition community to explore recognition of other scripts such as Arabic and various Indic scripts. One difficulty with the Arabic script is the number and position of diacritic marks associated to Arabic characters.
 
%\end{document}
