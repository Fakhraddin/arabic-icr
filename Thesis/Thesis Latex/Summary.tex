%\documentclass[12pt,english]{report}
%\usepackage{mathptmx}
%\renewcommand{\familydefault}{\rmdefault}
%\usepackage[T1]{fontenc}
%\usepackage[latin9]{inputenc}
%\usepackage[a4paper]{geometry}
%\setcounter{secnumdepth}{2} % Changed from 3 to 2. 0-chapter 1-section 2-subsection 
%\setcounter{tocdepth}{2} % Changed from 3 to 2. 0-chapter 1-section 2-subsection 
%\setlength{\parskip}{\medskipamount}
%\setlength{\parindent}{0pt}
%\usepackage{verbatim}
%\usepackage{pdfpages}
%\usepackage{graphicx}
%\usepackage{subfig} %% This package has to be here
%\usepackage{setspace}
%\usepackage{arabtex}
%\usepackage[numbers]{natbib}
%\usepackage{nomencl}
%\usepackage{amsthm}
%\usepackage{amsmath}
%\usepackage{amsfonts}
%\usepackage{paralist}
%\usepackage{etoolbox}
%\newtoggle{edit-mode}
%\toggletrue{edit-mode}  
%%%\toggletrue{edit-mode}
%\iftoggle{edit-mode}{
%\geometry{verbose,tmargin=2cm,bmargin=2cm,lmargin=2cm,rmargin=6cm,headheight=1cm,headsep=1cm,footskip=1cm, marginparwidth=5cm}
%}{
%\geometry{verbose,tmargin=2cm,bmargin=2cm,lmargin=2cm,rmargin=2cm,headheight=1cm,headsep=1cm,footskip=1cm}
%}
%
%
%\begin{document}

\chapter{Summary and Future Work}
\label{chap:summary}

\section{Summary}
\label{sec:summary}

The cursiveness of the Arabic script, prima facie, requires delaying the launch of the recognition process until the completion of the word scribing. 
However, in this study, we question the necessity of the requirement by demonstrating the feasibility of determining the position of the segmentation points while the stroke is being written; and by doing so, accelerating the recognition process.

The proposed recognition-based segmentation method is performed in the stroke level enabling analysis task related to segmentation and recognition of the stroke into letters to be performed while the stroke is being written. 
The first stage, which requires the most calculation effort, is performed whilst the stroke is being written and involves over-segmentation of the stroke using Morphological features and the sub-strokes induced by the segmentation points are classified using a $k$-NN letters classifier.
The classification information was saved in a scoring matrix.
The second stage contains filtering of the topologically invalid segmentation points. 
The last stage involves several segmentation points selections algorithm which determines the best subset of segmentation points.

The real-time nature of the segmentation system requires implementing a fast character classifier.
In this thesis we propose a high-performance approach for Arabic handwritten characters classification.
Using an efficient EMD embedding into the wavelet coefficient domain, the feature vectors were translated into the $L_1$ space, facilitating fast computation of the EMD approximation by calculating the Manhattan distance between the embedded vectors.
Fast retrieval of the $k$ nearest neighbours of a given query character was then possible.
A combination of PCA and LDA was employed to reduce the dimensionality of the sparse vectors generated by the embedding function, and DTW was used to refine the similarity scoring of the candidates.
We have evaluated the contribution of each stage in the classification flow to the performance and accuracy of the system.

The system have achieved promising results without using contextual information.

\section{Future Directions}
\label{sec:future_directions}

In future work, we will expand the process by adding a later holistic word part recognizer that exploits the proposed segmentation and the scoring information to inspect a limited number of potential word parts.
Also, we are planning to allow the letter writing to be expand over multiple strokes.
In addition we will develop a word completion algorithm based on our real-time recognition technique.

In a future work, the additional strokes will be empl0yed to discriminate between letters having the same main body (i'jam).
We believe that this will significantly improve the segmentation and classification results, as the additional strokes can help decrease the potential letter set.

We are planning to enhance the segmentation points nomination by employing morphological learning algorithm for determining if the given point is a potential segmentation point based on the surrounding environment of the candidate point. 
In addition are planning to standardize the characters database extracted from the ADAB database.

%\end{document}
