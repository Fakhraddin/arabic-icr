%\documentclass[12pt,english]{report}
%\usepackage{mathptmx}
%\renewcommand{\familydefault}{\rmdefault}
%\usepackage[T1]{fontenc}
%\usepackage[latin9]{inputenc}
%\usepackage[a4paper]{geometry}
%\setcounter{secnumdepth}{2} % Changed from 3 to 2. 0-chapter 1-section 2-subsection 
%\setcounter{tocdepth}{2} % Changed from 3 to 2. 0-chapter 1-section 2-subsection 
%\setlength{\parskip}{\medskipamount}
%\setlength{\parindent}{0pt}
%\usepackage{verbatim}
%\usepackage{pdfpages}
%\usepackage{graphicx}
%\usepackage{subfig} %% This package has to be here
%\usepackage{setspace}
%\usepackage{arabtex}
%\usepackage[numbers]{natbib}
%\usepackage{nomencl}
%\usepackage{amsthm}
%\usepackage{amsmath}
%\usepackage{amsfonts}
%\usepackage{paralist}
%\usepackage{etoolbox}
%\newtoggle{edit-mode}
%\toggletrue{edit-mode}  
%%%\toggletrue{edit-mode}
%\iftoggle{edit-mode}{
%\geometry{verbose,tmargin=2cm,bmargin=2cm,lmargin=2cm,rmargin=6cm,headheight=1cm,headsep=1cm,footskip=1cm, marginparwidth=5cm}
%}{
%\geometry{verbose,tmargin=2cm,bmargin=2cm,lmargin=2cm,rmargin=2cm,headheight=1cm,headsep=1cm,footskip=1cm}
%}
%
%
%\begin{document}

\chapter{Summary and Future Work}
\label{chap:summary}

\section{Summary}
\label{sec:summary}

The cursiveness of the Arabic script, prima facie, requires delaying the launch of the recognition process until the completion of the word scribing. 
However, in this study, we question the necessity of the requirement by demonstrating the feasibility of determining the position of the SPs while the stroke is being written; and by doing so, accelerating the recognition process.
The proposed segmentation method is performed in the strokes level enabling analysis task related to segmentation and recognition of the stroke into letters to be performed while the stroke is being written. 

Linear EMD embedding technique was used as a metric in the classification algorithms and showed promising results.
A composition of PCA followed by LDA was applied to achieve qualitative dimensionality reduction. 
We have integrated fast data retrieval methods to boost the performance of the system. 
The system demonstrated superior high recognition precision and low response time.

In the second part of the thesis we proposed a on-line approach for segmenting open-dictionary, Arabic handwritten script.
The segmentation is performed in the strokes level. 
Morphological features are employed to nominate potential SPs. 
The sub-strokes induced by the SPs are classified using a $k$-NN letters classifier, and the classification information is saved in a scoring matrix.
The final segmentation points are then selected by finding the best-scored segmentation path in the scoring matrix. 
This is done using several proposed algorithms.

The first stage, which requires the most calculation effort, is performed whilst the stroke is being written and involves over-segmentation using topological and directional local features and recognition-based scoring of adjacent combinations of sub-strokes that induced by the proposed segmentation points. 
The second stage contains filtering of the topologically invalid segmentation points. 
The last stage involves several segmentation points selections algorithm which determines the best subset of segmentation points.

The system have achieved promising results without using contextual information.

%%%%%%%%%%%%%%%%%%%%%%%%%%%%%%%%%%%%%%%%%%%%%%%%%%%%%%%
\newpage{}
%%%%%%%%%%%%%%%%%%%%%%%%%%%%%%%%%%%%%%%%%%%%%%%%%%%%%%%

\section{Future Directions}
\label{sec:future_directions}
In future direction we would like to rid of all the limitation mentioned in the limitation section.
We have several directions we would like to develop. In this work we did not consider the additional strokes written and we tried to recognize only the main body of the written text. In future work we would like to be able to recognize the whole letter and use the additional strokes to be able to identify the exact letter written.
The following enhancement will be introduced to the process in a future work: 1. Since the embedding is done to the L1 space, we will try to use L1 dimensionality reduction techniques such as L1-PCA and L1-LDA. 2. We can add Metric Learning techniques such as LMNN to improve our recognition rate. 3. In this work all shapes that belong to the same letter are ascribed to the same cluster although it might look totally different, in our future work we will create more that cluster to the same letter to represent the main shapes that are common to describe a letter.[??] 4. The use of more advanced clustering techniques to reduce the number of samples in the training set and which will accelerate the classification process.
Also, we are planning to develop a word completion algorithm based on our ongoing recognition technique. The word completion feature is actually  will suggest the user to automatically complete the word the user has intended to write, based on the first letters written by him. 
We also plan to connect a word parts or words database to the recognition system, which will certainly improve the recognition rate dramatically. However it will limit the recognized word/word parts to the dictionary.

The system has demonstrated promising results.

In future work, we will expand the process by adding a later holistic WP recognizer that exploits the proposed segmentation and the scoring information to inspect a limited number of potential WPs.\\

Also we plan to test this approach on other cursive handwriting styles of languages other than Arabic. 

\begin{itemize}
\item The use of the temporal information of the written for both classification and segmentation. Since on-line handwritten script give us this information we better use it to enhance our system. 
\item use more sophisticated segmentation techniques and rules. i.e. instead of using the static rules to detect segmentation points.   
\item Standardize the letters database extracted from the ADAB database.
\end{itemize}

%Our approach successfully tackled many inherent difficulties in recognizing Arabic script such as segmentation of word parts, non-horizontal writing style, position-depended letter shaping and large word-part dictionary which has been overcome upon by combining both the analytic and the heuristic approach.

%\end{document}
