%%% LyX 2.0.5.1 created this file.  For more info, see http://www.lyx.org/.
%%% Do not edit unless you really know what you are doing.
%\documentclass[twoside,english]{report}
%\usepackage[T1]{fontenc}
%\usepackage[latin9]{inputenc}
%\setcounter{secnumdepth}{3}
%\setcounter{tocdepth}{3}
%\usepackage[active]{srcltx}
%\usepackage{verbatim}
%\usepackage{graphicx}
%\usepackage{setspace}
%\usepackage[numbers]{natbib}
%\doublespacing
%
%\makeatletter
%
%%%%%%%%%%%%%%%%%%%%%%%%%%%%%%% LyX specific LaTeX commands.
%\providecommand{\LyX}{L\kern-.1667em\lower.25em\hbox{Y}\kern-.125emX\@}
%%% Because html converters don't know tabularnewline
%\providecommand{\tabularnewline}{\\}
%
%\makeatother
%
%\usepackage{babel}
%\begin{document}

\chapter{Literature Review}
Automated recognition of text has been an active subject of research since the early days of computers. A 1972 survey cites nearly 130 works on the subject \cite{harmon1972automatic}. Despite the age of the subject, it remains one of the most challenging and exciting areas of research in computer science. In recent years it has grown into a mature discipline, producing a huge body of work.
Despite long standing predictions that handwriting, and even paper itself, would become obsolete in the age of the digital computer, both persist. Whilst the computer has hugely simplified the process of producing printed documents, the convenience of a pen and paper still makes it the natural medium for many important tasks.
A brief survey of students in any lecture theater will confirm the dominance of handwritten notes over those typing on laptops. However, the ease and convenience of having information in digital form provides a powerful incentive to find a way of quickly converting handwritten text into its digital equivalent.
The field of handwriting recognition can be split into two different approaches. The first of these, on-line, deals with the recognition of handwriting captured by a tablet or similar touch-sensitive device, and uses the digitized trace of the pen to recognize the symbol. In this instance the recognizer will have access to the x and y coordinates as a function of time, and thus has temporal information about how the symbol was formed. The second approach concentrates on the recognition of handwriting in the form of an image, and is termed off-line. In this instance only the completed character or word is available. It is this off-line approach that will be taken in this report.

{\color{blue}Printed Arabic text is like handwritten Latin text, such that connection of characters is an inherent property for Arabic script whether it is typed, printed or handwritten. Most of errors and deficiencies of Arabic recognition systems comes from the segmentation stages Various segmentation algorithms have been proposed in the literature. Given the vast number of papers published on OCR, it is impossible to include all the segmentation methods in this survey. (Zidouri, Sarfraz, Shahab, \& Jafri, 2005)}

\emph{Need to decide about what to write in this section. Should it contain literature review about every stage in the system? i.e. Pre-processing, segmentation, letters recognition, and post-processing.}

\section{Arabic On-line Recognition Approaches}

\subsection{The Analytic Approach}
\cite{burrow2004arabic}

\subsection{The Holistic Approach}
\cite{burrow2004arabic}

\section{Arabic Text Segmentation}
\emph{copy paste from the paper}

%\end{document}
