%\documentclass[12pt,english]{report}
%\usepackage{mathptmx}
%\renewcommand{\familydefault}{\rmdefault}
%\usepackage[T1]{fontenc}
%\usepackage[latin9]{inputenc}
%\usepackage[a4paper]{geometry}
%\setcounter{secnumdepth}{2} % Changed from 3 to 2. 0-chapter 1-section 2-subsection 
%\setcounter{tocdepth}{2} % Changed from 3 to 2. 0-chapter 1-section 2-subsection 
%\setlength{\parskip}{\medskipamount}
%\setlength{\parindent}{0pt}
%\usepackage{verbatim}
%\usepackage{pdfpages}
%\usepackage{graphicx}
%\usepackage{subfig} %% This package has to be here
%\usepackage{setspace}
%\usepackage{arabtex}
%\usepackage[numbers]{natbib}
%\usepackage{nomencl}
%\usepackage{amsthm}
%\usepackage{amsfonts}
%\begin{document}

\chapter{Wavelets and the Wavelet Transform}
\label{app:wavelets}

The topic of wavelet analysis is relatively a new. The wavelet transform, like the Fourier transform, is used in many fields of mathematics, physics and engineering. This concept usually requires a high level background in mathematics to be deeply comprehended. Quite a few articles and books were written on the subject, therefore, here we provide a short and intuitive introduction to the topic. The reader is referred to \cite{meyer1992wavelets} for a comprehensive discussion on the subject.\\ 

Recall that the Fourier transform provides a frequency related information of a given signal. While it reveals what frequencies can be found in the signal, it does not indicate on what time, exactly, these frequency components exist. In cases where the signal is stationary, this information is irrelevant. However, when used for non-stationary signals, if we want to know what frequency component occurs at what time (interval), then Fourier transform is not the best transform to use. Wavelet transform provides us with the time-frequency representation of the signal and thus gives us information about both the time and frequencies.\\

Wavelets may be seen as small waves $\psi(t)$, which oscillate few times, but unlike the harmonic waves ($sin(nx)$ and $cos(nx)$ on $(-\pi,\pi]$) must die out to zero as $t \rightarrow \pm\infty$. The most applicable wavelets are those that die out after a few oscillations on a finite interval $[a,b)$. \\

The fundamental idea of the wavelet transform is to convert a signal into a series of similar shapes wavelets that vary in location and expansion, and by doing so, to provide a way for analyzing waveforms bounded in both frequency and duration. 
In order to analyze a given signal function $f(t)$ using the wavelet transform, one needs to create a mathematical structure that vary in scale. i.e., to construct a function, shift it by some amount, and change its scale. Repeat the procedure to obtain a new approximation. This is usually refereed to as a \emph{multi-resolution analysis}.\\ 

Analyzing a non-stationary signal requires determining its behavior at any individual event. Multi-resolution analysis provides one means to do this. A multi-resolution analysis decomposes a signal into a smoothed version of the original signal and a set of detail information at different scales. Once we have decomposed a signal this way, we may analyze the behavior of the detail information across the different scales. This is done to reveal information that is not obvious in the original time domain function, to allow signals to be stored efficiently and facilitate better approximation of real-world signals. \\

For a given time-domain signal function $f(t)$ and a wavelet function $\psi(t)$, Equation \ref{eq:wavelet_decomposition} formally defines the wavelet transform.
\begin{equation}
\Psi^{\psi}_{f}(\tau, s)=\frac{1}{\sqrt{|s|}}\int{f(t)\psi^*(\frac{t-\tau}{s})dt}
\label{eq:wavelet_decomposition}
\end{equation}
As seen in Equation \ref{eq:wavelet_decomposition} above, the transformed signal is a two variables function, $\tau$ and $s$, the translation and scale parameters respectively. $\psi(t)$ is called the mother wavelet.\\

The scale parameter $s$ can be thought of as exactly the scaling in maps. High scales correspond to a non-detailed high level view (of the signal), and low scales correspond to a detailed low level view. Similarly, in terms of frequency, low frequencies (high scales) correspond to a global information of a signal (that usually spans the entire signal), whereas high frequencies (low scales) correspond to a detailed information of a hidden pattern in the signal.\\

A function $f \in \mathds{R}$ can be expressed in terms of wavelet series as follows:
\begin{equation}
f(x)=\sum\limits_{k}{f_k \phi(x-k)}+\sum\limits_{k}{f_{j,k}\psi_{j,k}(x)}
\end{equation}
where $\phi$ is the scaling function and $\psi$ is the wavelet. $k$ runs through all integer and represents shifts.\\

%\bibliographystyle{plainnat}
%\bibliography{references}
%
%\end{document}
